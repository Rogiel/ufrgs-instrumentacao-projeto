\documentclass[12pt,a4paper]{instrumentacao}

\usepackage{listings}

\graphicspath{
	{../Resources/Images/}
	{../Resources/Mathematica/images/}
	{../Resources/MATLAB/images/}
}

\title{Central meteorológica: Dispositivo para medição de precipitação, de temperatura e de velocidade do vento}
\author{Rogiel Sulzbach \and Rodrigo de Castro Silveira \and Yi Chen Wu}
\startdate{18 de abril de 2016}
\finishdate{27 de junho de 2016}
\emails{
	\emailaddress{R.J.S.}{rogiel@rogiel.com},
	\emailaddress{R.C.S.}{csilveira.rodrigo@gmail.com} e
	\emailaddress{Y.C.}{yichenpoa@gmail.com}
}
\resume{Este trabalho visa o desenvolvimento do trabalho final da disciplina cujo objetivo é construir uma estação meteorológica que permite fazer algumas medições que, através dos dados recolhidos em um determinado período, contribuem para fazer a previsão de tempo e também a caracterização do clima. Para tanto, será desenvolvido instrumentos (ou sensores eletrônicos) de medição que registram as variáveis meteorológicas e climáticas. Como uma estação meteorológica possui muitos instrumentos de medição, para este trabalho, serão feitos apenas alguns deles, tais como o termômetro para medir a temperatura, o anemômetro para medir a velocidade do vento e o pluviômetro para medir a precipitação pluviométrica}
\abstract{This class final project objective is to build a meteorological station which allows to collect data over time whose measurements contribute to weather and to climate characterization. For such, sensors and electronic sensors will be developed which will register meteorological variables. As a meteorological station has several instruments, only a subset of those will be implemented on this project, these include a thermometer to measure temperatures, anemometer to measure the wind speed and a rain gauge to measure the rainfall.}
\keywords{estação meteorológica; instrumentos de medição; temperatura; quantidade de chuva; velocidade de vento.}
\institute{Universidade Federal do Rio Grande do Sul, Departamento de Engenharia Elétrica, Curso de Engenharia Elétrica, Instrumentação A, Prof. Dr. Alexandre Balbinot}

\headertext{Proposta de Projeto}

\begin{document}
\maketitle

\chapter{Introdução}
De acordo com o livro-texto da disciplina \textit{ENG04457 - Instrumentação A} \cite{livro-texto}, em seu prefácio, expõe que "hoje o mundo não escreve: digita. Internet, iPod, celulares, pen-drives... Toda essa revolução na sociedade criou novas habilitações dentro das engenharias, como, por exemplo, as engenharias de computação, de software e de tecnologia da informação. O analógico aparentemente tornou-se obsoleto, tirando a atenção de disciplinas relacionadas, como instrumentação, sensores e transdutores -- um equívoco ocorrido em função da compreensão superficial do que está por trás desta revolução tecnológica que estamos vivendo. Porém, o som, a imagem e diversos outros fenômenos que nos cercam são entes analógicos. Antes de virar bytes, o mundo é captado analogicamente". É com base nesta necessidade do domínio da captação do mundo analógico que a disciplina corretamente propõe a elaboração, planejamento e montagem de um projeto experimental utilizando os conceitos vistos em sala de aula.

O grupo, orientado por algumas indicações do professor da disciplina, optou pela elaboração do projeto de uma \textit{Central Meteorológica} capaz de executar medições de precipitação (chuva), de temperatura e de velocidade do vento. De acordo com a enciclopédia eletrônica livre \textit{Wikipedia}\cite{estacao}, "uma estação meteorológica é um local onde são recolhidos dados para análise do tempo meteorológico. Encontram-se equipadas com instrumentos (ou sensores eletrônicos) de medição e registro das variáveis meteorológicas/climáticas. Os seus dados são utilizados para a previsão do tempo e para a caracterização do clima, pelo que também podem ser designadas por estações climatológicas".

O projeto desenvolvido pelo grupo fará a captação e tratamento dos dados de temperatura, velocidade do vento e volume de precipitação. Assim, serão projetados os seguintes instrumentos:

\begin{itemize}
	\item \textbf{Termômetro}: para medir a temperatura;
	\item \textbf{Anemômetro}: para medir a velocidade do vento;
	\item \textbf{Pluviômetro}: para medir a precipitação pluviométrica.
\end{itemize}

\chapter{Metodologia Experimental}
A seguir serão apresentadas as especificações de cada um dos instrumentos a serem projetados e implementados para a montagem da Central Meteorológica, objeto do projeto final da disciplina.

\section{Termômetro}
Para o termômetro utilizaremos um sensor do tipo RTD a fim de medir a temperatura do ar ambiente onde a estação está instalada.

\section{Anemômetro}
O anemômetro é um instrumento que mede a velocidade do vento horizontal. Os anemômetros de copo são o tipo padrão de anemômetro, pois são robustos e resistentes aos ventos oblíquos causados por mastros e travessas, e este tipo será utilizado no projeto em questão, a exemplo da Figura \ref{fig:anemometro}

\begin{figure}[h]
	\centering
		\includegraphics{Figuras/anemometro.png}
	\caption{Exemplo de anemômetro de copo}
	Fonte: \url{https://www.google.com.br/url?sa=i&rct=j&q=&esrc=s&source=images&cd=&cad=rja&uact=8&ved=&url=http\%3A\%2F\%2Fwww.seinstrumentos.com.br\%2Fanemometro.html&psig=AFQjCNGjijn5ng5hpmUAcBq4F_TVvceWUw&ust=1461032511805601}
	\label{fig:anemometro}
\end{figure}

Como forma de medir a velocidade de rotação do anemômetro, será utilizado um sensor de efeito \textit{Hall} associado com um ímã. Assim, toda vez que o ímã passar pelo sensor de efeito \textit{Hall}, este deixa passar corrente, podendo assim ser contabilizado o número de voltas em um determinado período de análise.


\section{Pluviômetro}
Segundo a Wikipedia \cite{pluviometro}, "o pluviômetro é um aparelho de meteorologia usado para recolher e medir, em milímetros lineares, a quantidade de líquidos ou sólidos (chuva, neve, granizo) precipitados durante um determinado tempo e local". O parâmetro determinado por esse instrumento chama-se \textit{índice pluviométrico}, que é o somatório da precipitação num determinado local durante um período de tempo estabelecido, medido em milímetros [$mm$]. Como o equipamento mensura a quantidade de chuva que precipita, é elementar para estudos meteorológicos e hidrológicos em conjunto com o sensor de temperatura.

A estrutura física do pluviômetro consiste em um sistema de captação de chuva, com uma área determinada, e um reservatório para coleta do material captado. Nesse reservatório há um sensor que possibilita medir a diferença do nível de fluído armazenado.

Optou-se, dentre os vários métodos de medição de nível, a medição por ultrassom. O princípio de funcionamento desse método é medir o tempo de eco de um sinal enviado por um transdutor piezoelétrico. O transdutor que transmite o sinal também pode fazer a leitura (ou podem ser módulos separados emissor-receptor). Quando esse transdutor atua como transmissor, é excitado com um sinal elétrico gerando uma onda mecânica. Quando atua como receptor, o transdutor recebe um sinal mecânico e converte-o em sinal elétrico \cite{livro-texto}. O tempo entre o sinal enviado e o sinal de eco corresponde ao dobro da distância entre o medidor e a superfície cujo nível está sendo medido, conforme exemplifica a equação \ref{eq:dist_ultrassom}.

\begin{equation}
	d=(vt)/2
	\label{eq:dist_ultrassom}
\end{equation}

Como a maneira de medir a quantidade de chuva é muito ampla, cabe aos membros do grupo investigar a melhor metodologia de medição apropriada para obter boas medidas.

<<<<<<< HEAD

=======
>>>>>>> origin/master
\chapter{Cronograma de Execução}

Na tabela \ref{tab:cronograma} está apresentado o cronograma de execução do projeto:

\begin{table}[H]
\centering
\caption{Cronograma de execução do projeto}
\label{tab:cronograma}
\begin{tabular}{|c|l|}
\hline
\textit{\textbf{Prazo}} & \multicolumn{1}{c|}{\textit{\textbf{Atividades}}}               \\ \hline
18/abr                  & Entrega da Proposta do Projeto Final                            \\ \hline
20/abr                  & Compra de sensores e componentes                                \\ \hline
27/abr                  & Construção mecânica do anemômetro e do pluviômetro              \\ \hline
04/mai                  & Levantamento das curvas dos sensores                            \\ \hline
18/mai                  & Design e montagem de placa de circuito impresso                 \\ \hline
25/mai                  & Testes e análise de incertezas dos sensores                     \\ \hline
01/jun                  & Finalização do Anteprojeto                                      \\ \hline
06/jun                  & Apresentação do Anteprojeto                                     \\ \hline
08/jun                  & Ajuste da interface gráfica                                     \\ \hline
15/jun                  & Ajustes, testes finais e finalização da documentação do projeto \\ \hline
22/jun                  & Ajustes, testes finais e finalização da documentação do projeto \\ \hline
27/jun                  & Apresentação do Projeto Final                                   \\ \hline
\end{tabular}
\end{table}

\begin{thebibliography}{9}
\bibitem{mathematica-numerial-precision} \url{https://reference.wolfram.com/language/tutorial/NumericalPrecision.html}, acessado em 16 de março de 2016

\bibitem{livro-texto}  Balbinot, A.; Brusamarello, V. J., Instrumentação e Fundamentos de Medida - Vol.1 - 2ª Ed. Rio de Janeiro: LTC, 2014.

\bibitem{estacao} \url{https://pt.wikipedia.org/wiki/Esta\%C3\%A7\%C3\%A3o_meteorol\%C3\%B3gica}, acessado em 16/04/2016.

\bibitem{pluviometro} \url{https://pt.wikipedia.org/wiki/Pluvi\%C3\%B4metro}, acessado em 16/04/2016.
\bibitem{pluviometro-1} \url{http://www.agsolve.com.br/dicas-e-solucoes/como-funciona-o-pluviometro}, acessado em 16/04/2016.



%\bibitem{ref1} Sobrenome, A.B.; Sobrenome, C.D. Title of the cited article. Journal Title 2007, 6, 100-110. 
%\bibitem{ref2} Balbinot, A.; Brusamarello, V.J.. Title of the cited article. Journal Title 2007, 6, 100-110. 
%\bibitem{ref3} Author, A.; Author, B. Title of the chapter. In Book Title, 2nd ed.; Editor, A., Editor, B., Eds.; Publisher: Publisher Location, Country, 2007; Volume 3, pp. 154-196.
%\bibitem{ref4} Author, A.; Author, B. Book Title, 3rd ed.; Publisher: Publisher Location, Country, 2008; 
%pp. 154-196.

\end{thebibliography}

\end{document}
